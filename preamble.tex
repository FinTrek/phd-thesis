%!TEX root = main.tex
\usepackage{algorithm}
\usepackage{algorithmicx}
\usepackage{algpseudocode}
\usepackage{amsmath}
\usepackage{amssymb}
\usepackage{amsthm}
\usepackage{appendix}
\usepackage{array} % for \extrarowheight
\usepackage{booktabs}
\usepackage{bm} % for \bm
\usepackage{cancel}
\usepackage{colortbl} % For nice color in tables
\usepackage{enumerate}
\usepackage{fancyhdr}
\usepackage[perpage]{footmisc} %pour recommencer ‡ indicer les notes de bas de pages ‡ 1 ‡ chaque page
\usepackage{float} % for [H]
\usepackage{framed} % to make boxes
\usepackage{graphicx}
\usepackage{lastpage}
\usepackage{listings}
\usepackage{lmodern}% better font than default
\usepackage{mathtools}
\usepackage{mathrsfs} % for mathscr
\usepackage{multicol}
\usepackage{multirow}
\usepackage{natbib}
\usepackage{relsize} % for \mathsmaller
\usepackage{setspace}
\usepackage{subfig}
\usepackage{tabularx}
\usepackage{tikz} % for drawing circles and lines (\sbullet & \ginv)
\usepackage{upgreek} % for uppi
\usepackage{url}
\usepackage{verbatim}
\usepackage{xcolor}
\usepackage{xspace}


\onehalfspacing

% Load images from subdirectory.
\graphicspath{{images/}}

\newcommand*{\FT}[1]{\mathcal{F}\{#1\}} % Fourier transform
\newcommand*{\FTsub}[1]{\mathcal{F}_{#1}} % Fourier transform subscript


% Probability symbols.
\DeclareMathOperator{\Prob}{{\mathbb P}}
\DeclareMathOperator{\Exp}{{\mathbb E}}
\DeclareMathOperator{\Cov}{\mathbb{C}\mathrm{ov}}
\DeclareMathOperator{\Var}{\mathbb{V}\mathrm{ar}}
\DeclareMathOperator{\Q}{\mathbb Q}
\DeclareMathOperator{\Corr}{\mathbb{C}\mathrm{orr}}

% Distributions.
\newcommand*{\UnifDist}{\mathsf{Uniform}}
\newcommand*{\ExpDist}{\mathsf{Exponential}}
\newcommand*{\GammaDist}{\mathsf{Gamma}}
\newcommand*{\ErlangDist}{\mathsf{Erlang}}
\newcommand*{\NormDist}{\mathsf{Normal}}
\newcommand*{\LNDist}{\mathsf{Lognormal}}
\newcommand*{\SLNDist}{\mathsf{SumLognormal}}

\newcommand*{\EllDist}{\mathsf{Elliptical}}
\newcommand*{\IGDist}{\mathsf{InverseGaussian}}
\newcommand*{\LaplaceDist}{\mathsf{Laplace}}
\newcommand*{\WeibullDist}{\mathsf{Weibull}}
\newcommand*{\ParetoDist}{\mathsf{Pareto}}

\newcommand*{\ClaytonCop}{\mathsf{Clayton}}
\newcommand*{\GumbelCop}{\mathsf{GumbelHougaard}}
\newcommand*{\FrankCop}{\mathsf{Frank}}

\newcommand*{\BinomialDist}{\mathsf{Binomial}}
\newcommand*{\PascalDist}{\mathsf{Pascal}}
\newcommand*{\PoissonDist}{\mathsf{Poisson}}

\newcommand*{\GMDA}{\mathsf{GMDA}}

% Limits and distributional stuff.
\newcommand*{\convas}{\xrightarrow{\:\smash{\mathrm{a.s.}}\:}}
\newcommand*{\convdistr}{\xrightarrow{\:\smash{\mathcal{D}}\:}}
\newcommand*{\eqdistr}{\stackrel{\smash{\mathcal{D}}}{=}}
\newcommand*{\approxdistr}{\mathrel{\dot\sim}}

\newcommand*{\NormCdf}{\mathrm{\Phi}}
\newcommand*{\iidDist}{\overset{\smash{\mathrm{iid}}}{\sim}}
\newcommand*{\indDist}{\overset{\smash{\mathrm{ind}}}{\sim}}

% Sets of numbers.
\newcommand*{\NL}{\mathbb{N}_+}
\newcommand*{\RL}{\mathbb{R}}
\newcommand*{\CL}{\mathbb{C}}
\newcommand*{\NZ}{\mathbb{N}_0}
\newcommand*{\ZL}{\mathbb{Z}}

% Roman versions of things.
\newcommand*{\dd}{\mathop{}\!\mathrm{d}}
\newcommand*{\e}{\mathrm{e}}
\newcommand*{\sign}{\mathrm{sign}}
\newcommand*{\diag}{\mathrm{diag}}
\DeclareMathOperator*{\argmax}{arg\,max}
\DeclareMathOperator*{\argmin}{arg\,min}
\DeclareMathOperator*{\supp}{Supp}
\DeclareMathOperator{\csch}{csch}

% Vector functions
\newcommand*{\ve}{\mathbf{e}} % vector e
\DeclareMathOperator{\vlog}{\mathbf{log}}	% vector log

% Indicator
\DeclarePairedDelimiterXPP{\ind}[1]{\mathbb{I}}{\{}{\}}{}{#1}

% Integral transforms (pjl: TODO finish this off)
\newcommand*{\Laplace}{\mathscr{L}}
\newcommand*{\LT}[1]{\mathscr{L}\{#1\}}
\newcommand*{\LTsub}[1]{\mathscr{L}_{#1}}
\newcommand*{\Lp}{\mathcal{L}} % e.g. L^2 loss.
\newcommand*{\PGF}{\mathcal{G}}

% Financial stuff.
\newcommand*{\VaR}{\mathrm{VaR}}

% Shortcuts to fix annoying defaults in Latex.
\renewcommand{\hat}[1]{\widehat{#1}}
\renewcommand{\tilde}[1]{\widetilde{#1}}
\renewcommand{\epsilon}{\varepsilon}
\renewcommand{\pi}{\uppi}

% Special symbols.
\newcommand*{\defeq}{:=}
\newcommand*{\defeqr}{=:}
\newcommand*{\tr}{\top}
\newcommand*{\ih}{\mathrm{i}}
\newcommand*{\oh}{{\mathrm{o}}}
\newcommand*{\Oh}{{\mathcal{O}}}
\newcommand*{\comp}{\mathsf{c}} % complement
\newcommand*{\argdot}{{}\cdot{}} % for functions like f(.)
\newcommand*{\BigBinom}[2]{\Bigl( \genfrac{}{}{0pt}{}{#1}{#2} \Bigr)}

% Stuff I used in paper one.
\newcommand*{\LambertW}{\mathcal{W}}
\newcommand*{\Fset}[1]{\mathcal{F}_{\!\mathsmaller{#1}}}
\newcommand*{\sbullet}{\tikz[baseline=0] \fill (0,0.05) circle (0.035);}
\newcommand*{\ms}[1]{\mathsmaller{#1}}

% Norm and absolute value signs.
\newcommand*{\card}[1]{\lvert #1\rvert}
\newcommand*{\abs}[1]{\lvert #1\rvert}
\newcommand*{\norm}[1]{\lVert #1\rVert}


\newcommand*{\alg}[1]{Algorithm~\ref{#1}}
\newcommand*{\fig}[1]{Figure~\ref{#1}}
\newcommand*{\thrm}[1]{Theorem~\ref{#1}}
\newcommand*{\cor}[1]{Corollary~\ref{#1}}
\newcommand*{\prop}[1]{Proposition~\ref{#1}}

\newcommand*{\fft}{f.f.t.\xspace}

% Regarding theorem numbering.
\theoremstyle{plain} % the default, bold header, italic body
\newcounter{thm}
\newtheorem{theorem}[thm]{Theorem}
\newtheorem{lemma}[thm]{Lemma}
\newtheorem{proposition}[thm]{Proposition}
\newtheorem{corollary}[thm]{Corollary}
\newtheorem{pseudoalgorithm}[thm]{Algorithm} % For the long detailed algorithm from the SLN LT paper.
\newtheorem{definition}[thm]{Definition}
\newtheorem{test}[thm]{Test}
\theoremstyle{definition} % Non-italics body
\newtheorem{example}[thm]{Example}
\newtheorem{remark}[thm]{Remark}
\newtheorem{assumption}[thm]{Assumption}

% Margins and text.
\setlength{\textwidth}{16cm}
\setlength{\oddsidemargin}{0pt}
\setlength{\evensidemargin}{0pt}
\setlength{\parskip}{3mm}
\setlength{\parindent}{0mm}

\numberwithin{thm}{chapter}

% \times with less space on left/right (for tables).
\newcommand*{\stimes}{{\mkern-2mu\times\mkern-2mu}}

\newcommand*{\remQED}{{\mbox{\, \vspace{3mm}}} \hfill \mbox{\tikz \draw [rotate=45, rounded corners=0.01pt] (0,0) rectangle (5.5pt,5.5pt);}}

\newcommand*{\cH}{{\mathcal H}}
\newcommand*{\cHb}{\overline{\mathcal H}}
\newcommand*{\ch}{{\mathcal h}}
\newcommand*{\cW}{{\mathcal W}}
\newcommand*{\cWb}{\overline{\mathcal W}}
\newcommand*{\cw}{{\mathcal w}}


\newcommand*{\ginv}{
  \tikz[baseline=-0.4ex] \draw[<-] (0,0)--(0.4em,0);
}


\newcommand*{\pinv}{[-1]}
\newcommand*{\alphaMax}{\overline{\alpha}}
\newcommand*{\Ftail}{\mkern 1mu\overline{\mkern-1mu{F}\mkern+1.25mu}}
\newcommand*{\indep}{\perp}
\newcommand*{\for}[1]{\,,\qquad \text{for } #1}

\definecolor{Gray}{gray}{0.9}


% All the bold symbols and letters.
\usepackage{etoolbox}
\newcommand\Bolder[1]{
  \ifcsundef{bf#1}{
    \csdef{bf#1}{\bm{#1}}
  }{
    \ERROR
  }
}
\newcommand\BolderS[1]{
  \ifcsundef{bf#1}{
    \csdef{bf#1}{\bm{\csname #1\endcsname}}
  }{
    \ERROR
  }
}

\forcsvlist\Bolder{a,b,c,e,h,m,q,r,t,u,v,w,x,y,z}
\forcsvlist\Bolder{A,B,C,D,F,H,I,S,U,W,X,Y,Z}
\forcsvlist\BolderS{alpha,beta,delta,gamma,omega,Lambda,Phi,mu,sigma,Sigma,theta,Theta}

\newcommand*{\bfzero}{\bm{0}}
\newcommand*{\bfone}{\bm{1}}
\newcommand*{\bfeye}{\bfI}

\newcommand{\Asymf}{\mathfrak{f}_S}
\newcommand{\AsymF}{\mathfrak{F}_S}
\newcommand{\AsymFTail}{\overline{\mathfrak{F}}_S}

\newcommand{\AsymfT}{\mathfrak{f}_T}
\newcommand{\AsymFT}{\mathfrak{F}_T}
\newcommand{\AsymFTailT}{\overline{\mathfrak{F}}_T}

\newcommand{\AsymfTrun}{\mathfrak{f}_{S|S>\gamma}}
\newcommand{\AsymFTrun}{\mathfrak{F}_{S|S>\gamma}}
\newcommand{\AsymFTailTrun}{\overline{\mathfrak{F}}_{S|S>\gamma}}